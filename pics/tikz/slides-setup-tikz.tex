\PassOptionsToPackage{subsection=false}{beamerouterthememiniframes} % Omit bar for the subsections
\documentclass[hyperref={colorlinks=true},10pt,aspectratio=43]{beamer} %handout
\setbeamersize{text margin left=0pt, text margin right=0pt}

%% Slide title and information
\title{{\small \textcolor{white}{GRAD-C6-2001} \vspace{.5cm}\\ \huge \textbf{\textcolor{white}{Statistics II:} \\ \textcolor{white}{Statistical Modeling}\\ \textcolor{white}{and Causal Inference}}}}
\author{\textcolor{white}{Simon Munzert }\\ \textcolor{white}{Hertie School}}
\institute[\textsc{\textcolor{white}{Simon Munzert}}]{}
\date{~}

% Theme
\usetheme{metropolis}
\setbeamertemplate{frame numbering}[none]
\setbeamercolor{background canvas}{bg=white}

% independent / notindependent sign
\makeatletter
% Taken from http://ctan.org/pkg/centernot
\newcommand*{\centernot}{%
  \mathpalette\@centernot
}
\def\@centernot#1#2{%
  \mathrel{%
    \rlap{%
      \settowidth\dimen@{$\m@th#1{#2}$}%
      \kern.5\dimen@
      \settowidth\dimen@{$\m@th#1=$}%
      \kern-.5\dimen@
      $\m@th#1\not$%
    }%
    {#2}%
  }%
}
\makeatother

\newcommand{\independent}{\perp\mkern-9.5mu\perp}
\newcommand{\notindependent}{\not{\independent}}


% Myblock
\newenvironment{myblock}[3]{%
  \setbeamercolor{block body}{#2}
  \setbeamercolor{block title}{#3}
  \begin{block}{#1}}{\end{block}}


% Theme fonts
\setmainfont{Fira Sans} 
\setsansfont{Fira Sans Light}

% Blocks
\setbeamertemplate{blocks}[rounded][shadow=false]
\setbeamercolor{block title}{bg=black!60,fg=white}
\setbeamercolor{block body}{bg=black!10}

% Footer
%\setbeamertemplate{frame footer}{My custom footer}


\usepackage[]{graphicx}
\usepackage[]{color}
%% maxwidth is the original width if it is less than linewidth
%% otherwise use linewidth (to make sure the graphics do not exceed the margin)
\makeatletter
\def\maxwidth{ %
  \ifdim\Gin@nat@width>\linewidth
    \linewidth
  \else
    \Gin@nat@width
  \fi
}
\makeatother

\definecolor{fgcolor}{rgb}{0.345, 0.345, 0.345}
\newcommand{\hlnum}[1]{\textcolor[rgb]{0.686,0.059,0.569}{#1}}%
\newcommand{\hlstr}[1]{\textcolor[rgb]{0.192,0.494,0.8}{#1}}%
\newcommand{\hlcom}[1]{\textcolor[rgb]{0.678,0.584,0.686}{\textit{#1}}}%
\newcommand{\hlopt}[1]{\textcolor[rgb]{0,0,0}{#1}}%
\newcommand{\hlstd}[1]{\textcolor[rgb]{0.345,0.345,0.345}{#1}}%
\newcommand{\hlkwa}[1]{\textcolor[rgb]{0.161,0.373,0.58}{\textbf{#1}}}%
\newcommand{\hlkwb}[1]{\textcolor[rgb]{0.69,0.353,0.396}{#1}}%
\newcommand{\hlkwc}[1]{\textcolor[rgb]{0.333,0.667,0.333}{#1}}%
\newcommand{\hlkwd}[1]{\textcolor[rgb]{0.737,0.353,0.396}{\textbf{#1}}}%

\newcommand{\raisedrule}[2][0em]{\leaders\hbox{\rule[#1]{1pt}{#2}}\hfill}
\newcommand{\rcodestart}{{\scriptsize \textcolor{darkgrey}{R code} \raisedrule[0.25em]{.5pt}}}
\newcommand{\rcodeend}{{\vspace{-.25cm} \scriptsize ~ \raisedrule[0.25em]{.5pt} \textcolor{darkgrey}{end}}}

\newenvironment{variableblock}[3]{%
  \setbeamercolor{block body}{#2}
  \setbeamercolor{block title}{#3}
  \begin{block}{#1}}{\end{block}}
  

\usepackage{framed}
\makeatletter
\newenvironment{kframe}{%
 \def\at@end@of@kframe{}%
 \ifinner\ifhmode%
  \def\at@end@of@kframe{\end{minipage}}%
  \begin{minipage}{\columnwidth}%
 \fi\fi%
 \def\FrameCommand##1{\hskip\@totalleftmargin \hskip-\fboxsep
 \colorbox{shadecolor}{##1}\hskip-\fboxsep
     % There is no \\@totalrightmargin, so:
     \hskip-\linewidth \hskip-\@totalleftmargin \hskip\columnwidth}%
 \MakeFramed {\advance\hsize-\width
   \@totalleftmargin\z@ \linewidth\hsize
   \@setminipage}}%
 {\par\unskip\endMakeFramed%
 \at@end@of@kframe}
\makeatother

\makeatletter
\AtBeginPart{%
  \addtocontents{toc}{\protect\beamer@partintoc{\the\c@part}{\beamer@partnameshort}{\the\c@page}}%
}
%% number, shortname, page.
\providecommand\beamer@partintoc[3]{%
  \ifnum\c@tocdepth=-1\relax
    % requesting onlyparts.
    \makebox[6em]{} \textcolor{green!30!blue}{\hyperlink{#2}{#2}}
    \par
  \fi
}
\define@key{beamertoc}{onlyparts}[]{%
  \c@tocdepth=-1\relax
}
\makeatother%

\newcommand{\nameofthepart}{}
\newcommand{\nupart}[1]%
    {   \part{#1}%
        \renewcommand{\nameofthepart}{#1}%
        \frame{\partpage \hypertarget{\nameofthepart}{}}%
    }

\definecolor{shadecolor}{rgb}{.97, .97, .97}
\definecolor{messagecolor}{rgb}{0, 0, 0}
\definecolor{warningcolor}{rgb}{1, 0, 1}
\definecolor{errorcolor}{rgb}{1, 0, 0}
\newenvironment{knitrout}{}{} % an empty environment to be redefined in TeX

\ifdefined\knitrout
  \renewenvironment{knitrout}{\begin{scriptsize}\setlength{\topsep}{-1mm}}{\end{scriptsize}}
\else
\fi


\usepackage[english]{babel}
\usepackage[babel,german=quotes]{csquotes}
\usepackage{graphicx}
\usepackage{color}
\usepackage{xcolor}
\usepackage{multirow}
\usepackage{multicol}
\usepackage{tabularx}
\usepackage{colortbl}
\usepackage{booktabs}
\usepackage{MnSymbol}
\usepackage[absolute,overlay]{textpos}
\usepackage{listings}
\usepackage{ulem}
\usepackage{url}
\urlstyle{sf}
\usepackage{hyperref}
\hypersetup{
  pdfauthor={...},
  pdftitle={...},
  pdfsubject={...},
  linkcolor=black,
  urlcolor=darkblue,
  citecolor=darkblue
}
\usepackage{pdfpages}
\usepackage[authoryear,round]{natbib}
\usepackage[symbol]{footmisc}
\usepackage{ragged2e}

\usepackage{pgf,tikz}
\usetikzlibrary{positioning,arrows,decorations.markings,arrows.meta}

%----- Definitions of colors -----%
\definecolor{markfunction}{RGB}{0,0,160}
\definecolor{markargument}{RGB}{158,202,225}
\definecolor{markcode}{RGB}{0,0,160}
\definecolor{markpackage}{RGB}{160,0,0}
\definecolor{markfile}{RGB}{165, 0, 38}

\definecolor{darkgrey}{RGB}{70,70,70}
\definecolor{darkblue}{RGB}{43,140,190}
\definecolor{darkgreen}{rgb}{0.0, 0.5, 0.0}
\definecolor{darkred}{rgb}{0.82, 0.1, 0.26}
\definecolor{darkorange}{RGB}{230,110,70}

\definecolor{lightgrey}{RGB}{220,220,220}
\definecolor{lightblue}{RGB}{158,202,225}
\definecolor{lightgreen}{rgb}{0.67, 0.88, 0.69}
\definecolor{lightred}{rgb}{1.0, 0.71, 0.76}
\definecolor{lightorange}{RGB}{255,230,150}

\definecolor{niceblue}{RGB}{0,36,93}
\definecolor{nicered}{RGB}{198,39,54}



% highlight macro
\usepackage{soul}
\DeclareRobustCommand{\hlnicered}[1]{{\sethlcolor{nicered}\hl{#1}}}
\DeclareRobustCommand{\hlniceblue}[1]{{\sethlcolor{niceblue}\hl{#1}}}


%\newcommand{\hlnice}[1]{\textcolor{blue}{\textbf{\hlnicered{~#1~}}}}%
%\newcommand{\hlnicetwo}[1]{\textcolor{white}{\textbf{\hlniceblue{~#1~}}}}%

\newcommand{\addfn}[1]{\noindent\rule{8cm}{0.4pt}\\{\footnotesize #1}}%

\newcommand{\hlnice}[1]{\textcolor{white}{\textbf{\colorbox{nicered}{#1}}}}
\newcommand{\hlnicetwo}[1]{\textcolor{white}{\textbf{\colorbox{niceblue}{#1}}}}



%----- Definitions of columntypes -----%
\newcolumntype{N}{>{\scriptsize}c}
\newcolumntype{M}{>{\scriptsize}l}
\newcolumntype{C}[1]{>{\centering\arraybackslash}p{#1}}
\newcolumntype{R}[1]{>{\raggedleft\arraybackslash}p{#1}}
\newcolumntype{L}[1]{>{\raggedright\arraybackslash}p{#1}}
\newcommand{\HRule}{\rule{\linewidth}{0.25mm}}

%----- Definitions of macros in text -----%
\newcommand{\func}[1]{\textcolor{markfunction}{\texttt{#1}}}
\newcommand{\argu}[1]{\textcolor{markargument}{\texttt{#1}}}
\newcommand{\code}[1]{\textcolor{markcode}{\texttt{#1}}}
\newcommand{\pack}[1]{\textcolor{markpackage}{\texttt{#1}}}
\newcommand{\file}[1]{\textcolor{markfile}{\textit{#1}}}
\newcommand{\R}{\textsf{R}}


%----- Beamer style issues -----%

% Change bullet style
% \useinnertheme{circles}
% \newcommand{\myitem}{\item[\textbullet]}
\beamertemplatenavigationsymbolsempty
\setbeamertemplate{itemize item}{\textbullet}
\def\Tiny{\fontsize{6pt}{6pt}\selectfont}



\lstdefinestyle{generalStyle} {
    basicstyle=\ttfamily\footnotesize\color{black},
    keywordstyle=\color{black},
    stringstyle=\color{darkblue},
    commentstyle=\color{black}
}




\lstloadlanguages{R}
%%listing environment for R code input
 \lstdefinestyle{Rinput} {
  name=input,
  language=R,
  style=generalStyle,
  numbers=left,  % where line numbers are displayed
  firstnumber=auto,
  stepnumber=1, % spacing between line numbers 
  aboveskip=0pt, % space above and
  belowskip=0pt, % below listing
  breaklines=true,      % line breaking of long lines.
  breakatwhitespace=false, % allows line breaks only at white space.
  breakindent=0pt,  % no indenting in second line
  breakautoindent=true, % apply intendation
  columns=flexible,    %
  keepspaces=true,
  showstringspaces=false,
  xleftmargin=0pt, % left indentation
  xrightmargin=0pt % right indentation
}

%%listing environment for R code output
 \lstdefinestyle{Routput} {
  name=output,
  language=R,
  basicstyle=\ttfamily\footnotesize\color{darkgrey},
  showstringspaces=false,
  aboveskip=2pt, % space above and
  belowskip=0pt, % below listing
  breaklines=true,      % line breaking of long lines.
  breakatwhitespace=false, % allows line breaks only at white space.
  breakindent=0pt,  % no indenting in second line
  breakautoindent=true, % apply intendation
  columns=flexible,    %
  keepspaces=true,
  xleftmargin=0pt, % left indentation
  xrightmargin=0pt % right indentation
}



